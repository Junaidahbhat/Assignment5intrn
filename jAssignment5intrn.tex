\documentclass[a4paper,12pt]{article}

\usepackage{gensymb}
\usepackage{tkz-euclide}
\usepackage{graphicx}
\usepackage{amsmath}

\begin{document}
\title{Assignment 5}
\author{Junaid Ahmad Bhat}
\date{\today}
\maketitle
\section*{\small Question:}
If AC = 7,$\angle$  A =60$^{\circ}$ and $\angle$B = 50$^{\circ}$, can you draw the triangle?
\section*{\small Solution:}
\begin{center}

\begin{tikzpicture}
    	
		
       \coordinate [label=left:$A$] (A) at (0,0);
		\coordinate [label=above:$B$] (B) at (3.225,5.58);
		\coordinate [label=right:$C$] (C) at (5.25,0);
	
		\draw (B) -- node[left,xshift=-1mm] {$\textrm{c}$} (A) -- node[below]
		{$\textrm{b}$} (C) -- node[right,,xshift=1mm] 
		{$\textrm{a}$} (B);
		  
		%\tkzMarkRightAngle[fill=black!10,size=0.8](A,B,C);
		  
\end{tikzpicture}\\
\hspace*{2cm}Rough Sketch\\
\end{center}

\vspace*{1cm}

To draw the Triangle with given conditions,
Let's first find the value of $\angle$C.\\

Using property, Sum of angles of triangle,\\
 
We have,\\

$\angle$ A + $\angle$ B + $\angle$ C = 180$^{\circ}$\\

60$^{\circ}$ + 50$^{\circ}$ + $\angle$ C =180$^{\circ}$  \hspace*{2cm} (given)\\

Therefore,$\angle$ C = 70$^{\circ}$.\\


\textbf{To draw the triangle,we need know the coordinates of all vertices.}\\

 Assuming starting vertix as A with coordinates as (0,0)\\

Therefore,\\
\hspace*{0.5cm} vertex C has coordinates as (7,0) \hspace*{3cm}(as given AC=7)\\

\textbf{Coordinates of vertix B:}\\

Actually vertix B is intersection point of lines AB and CB,so let's find it.\\

Eqn of line AB:y=mx+c;     \hspace*{2cm} (m=slope of line and c is constt.)\\

\hspace*{2.8cm}y=1.732x+c  \hspace*{2cm}  (as given $\angle$ A=60$^{\circ}$ $\Rightarrow$ m=tan(60$^{\circ}$))\\

\hspace*{0.5cm} value of c=0-1.732(0)=0 \hspace*{2cm}(as line AB passes through (0,0))\\

Therefore,\\

\hspace*{0.5cm}line AB:y=1.732x.  \hspace*{3cm}(1) \\

Similarly,\\

Eqn of line AB:y=mx+c;     \hspace*{2cm} (m=slope of line and c is constt.)\\

\hspace*{2.8cm}y=-2.727x+c  \hspace*{2cm}  (as  $\angle$ C=70$^{\circ}$ $\Rightarrow$ m=tan(110))\\

\hspace*{0.5cm} value of c=0+2.747(7)=0 \hspace*{2cm}(as line CB passes through (7,0))\\

Therefore,\\

\hspace*{0.5cm}line CB:y=-2.747x+19.23.   \hspace*{2.5cm}(2) \\

\pagebreak
\textbf{Finding Intersection point of line AB and CB}\\

Substuting,eqn(1) in eqn(2),we get\\

x=$\dfrac{19.23}{(1.73+2.727)}$\\

$\Rightarrow$ x=4.292.\hspace*{3cm} (3)\\

Substuting,eqn(3) in eqn(1),we get\\

y=1.73(4.292)\\

$\Rightarrow$ y=7.44.\hspace*{3cm} (4)\\

\textbf{Therefore,coordinates of vertex B are (4.292,7.44)}\\

Below figure 1 is possible triangle we can draw triangle with coordinates
\hspace*{0.5cm} as A(0,0), B(4.292,7.44) and C(7,0).\\

\vspace*{1cm} 
 
\begin{figure}[h]
\centering
\includegraphics[width=0.8\textwidth]{Assignment5intrn1}
\caption{Figure using python}
\end{figure}

\pagebreak

\section*{\small Question:}
Construct $\Delta$ PQR if PQ = 5, $\angle$Q = 105$^{\circ}$ and $\angle$R = 40$^{\circ}$.
\section*{\small Solution:}
\begin{center}

\begin{tikzpicture}
    	
		
       \coordinate [label=left:$P$] (P) at (0,0);
		\coordinate [label=below:$Q$] (Q) at (3.75,0);
		\coordinate [label=right:$R$] (R) at (4.61,3.225);
	
		\draw
        (Q) -- node[below,xshift=-1mm]
		{$\textrm{r}$} 
		(P) -- node[left,xshift=-1mm]
		{$\textrm{q}$} 
		(R) -- node[right,,xshift=1mm] 
		{$\textrm{p}$} (Q);
		  
		%\tkzMarkRightAngle[fill=black!10,size=0.8](A,B,C);
		  
\end{tikzpicture}\\
\hspace*{2cm}Rough Sketch\\
\end{center}

\vspace*{1cm}

To draw the Triangle with given conditions,
Let's first find the value of $\angle$P.\\

Using property, Sum of angles of triangle,\\
 
We have,\\

$\angle$ P + $\angle$ Q + $\angle$ R = 180$^{\circ}$\\

$\angle$ P  + 105$^{\circ}$ + 40$^{\circ}$ =180$^{\circ}$ \hspace*{2cm} (given) \\

Therefore,$\angle$ P  = 35$^{\circ}$.\\

\textbf{To draw the triangle,we need know the coordinates of all vertices.}\\

 Assuming starting vertix as P with coordinates as (0,0)\\

Therefore,\\
\hspace*{0.5cm} vertex Q has coordinates as (5,0) \hspace*{3cm}(as given PQ=5)\\

\textbf{Coordinates of vertix R:}\\

Actually vertix R is intersection point of lines PR and QR,so let's find it.\\

Eqn of line PR:y=mx+c;     \hspace*{2cm} (m=slope of line and c is constt.)\\

\hspace*{2.8cm}y=0.7x+c  \hspace*{2cm}  (as  $\angle$ P=35$^{\circ}$ $\Rightarrow$ m=tan(35$^{\circ}$))\\

\hspace*{0.5cm} value of c=0-0.7(0)=0 \hspace*{2cm}(as line PR passes through (0,0))\\

Therefore,\\

\hspace*{0.5cm}line PR:y=0.7x.  \hspace*{3cm}(1) \\

Similarly,\\

Eqn of line PR:y=mx+c;     \hspace*{2cm} (m=slope of line and c is constt.)\\

\hspace*{2.8cm}y=3.73x+c  \hspace*{2cm}  (as  $\angle$ Q=105$^{\circ}$ $\Rightarrow$ m=tan(75))\\

\hspace*{0.5cm} value of c=0-3.73(5)=0 \hspace*{2cm}(as line QR passes through (5,0))\\

Therefore,\\

\hspace*{0.5cm}line QR:y=3.73x-18.65.   \hspace*{2.5cm}(2) \\

\textbf{Finding Intersection point of line PR and QR}\\

Substuting,eqn(1) in eqn(2),we get\\

x=$\dfrac{18.65}{(3.73-0.7)}$\\

$\Rightarrow$ x=6.15.\hspace*{3cm} (3)\\

Substuting,eqn(3) in eqn(1),we get\\

y=0.7(6.15)\\

$\Rightarrow$ y=4.3.\hspace*{3cm} (4)\\

\textbf{Therefore,coordinates of vertex R are (6.15,4.3)}\\

Below figure 2 is given triangle with coordinates as P(0,0),Q(7,0) and  \hspace*{0.5cm}R(6.15,4.3).\\
 
 \vspace*{2cm}
 
\begin{figure}[h]
\centering
\includegraphics[width=0.8\textwidth]{Assignment5intrn2}
\caption{Figure using python}
\end{figure}

\vspace*{4cm}
\textbf{Note:}m (i,e slope)=tan(angle measured anti-clock wise from x-axis)

\end{document}